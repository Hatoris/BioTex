\documentclass[a4paper,12pt]{article}

\usepackage[utf8]{inputenc}
\usepackage[T1]{fontenc}
\usepackage[french, english]{babel} 
\usepackage[pdftex]{graphicx}
\usepackage[T1]{fontenc}
\usepackage[margin=2cm]{geometry}
\usepackage{amsmath}
\usepackage{appendix}
\usepackage{fancyhdr}
\usepackage[hidelinks]{hyperref}
\usepackage{cite}
\usepackage{doi}
\usepackage{setspace}
\usepackage{etoolbox} 
\usepackage{titlesec}
\usepackage{float}
\usepackage{caption}
\usepackage{hyperref}
\usepackage{booktabs}
\usepackage{adjustbox}
\usepackage[flushleft]{threeparttable}
\usepackage[french]{nomencl}
\usepackage{pythontex}
\usepackage{"C:/Users/Florian/OneDrive\space - \space Universite \space de \space Montreal/LaTeX/BioTex/BioTex/package/BioAbreviation"}

\newcommand{\PATH}[2]{"C:/Users/Florian/OneDrive - Universite de Montreal/LaTeX/BioTex/BioTex/#1/#2"}


\makenomenclature
\renewcommand{\nomname}{Liste des abréviations et des symboles}

\begin{document}
\title{}
\author{Florian Bernard}
\date{Le \selectlanguage{french}\today} 
\maketitle

\section{Introduction}

\section{Matériels}

\begin{pycode}
cells_infos = [
    ["Nom des cellules", "Passages", "Date d'isolation"],
    ["BMEC", "1", "2018-11-12"],
    ["bEnd3", "9", "-"],
    ["BMEC", "1", "2019-02-08"],

]
\end{pycode}

%\input{\PATH{material}{Cellules}}

\begin{pycode}
conso_infos = [
    ["Nom", "Concentration", "N° de catalogue", "N° de lot", "Compagnie"],
    ["DMEM", "-", "319-051-CL", "-", "Wisent"],
    ["PBS", "-", "311-010-CL", "-", "Wisent"],
]
\end{pycode}

%\input{\PATH{material}{Consommables}}

\section{Protocoles}

\begin{pycode}

nb_puits = 48 # nombre de puits à revêtir
vol_par_puit = 0.1 # 0.1 mL 100 uL 
Q_coll_par_puit = round(0.33 * 5, 2) # ug 5 ug/cm2 * 0.33 cm2

Ci = 1 #ug/uL
Cf = round(Q_coll_par_puit/(vol_par_puit * 1000), 4) # ug/uL
Vf = round(nb_puits * vol_par_puit * 1000 * 1.1, 2) # uL avec 10% pour les pertes
Vi = round(Cf * Vf / Ci, 2) # uL

start_incubation = "11h00"
end_incubation = "13h00"
TOTAL_incubation = ""

\end{pycode}

%\input{\PATH{protocole}{Transwell_collagen.tex}}

\begin{pycode}

nb_puit_cellule = 32
conc_cell_surf = 4e5 # cellule/cm2 
Vol_depos  = 100 # uL
Ci = 5.2e6 # cellule / mL
Vi = 1 # mL

nb_cell_par_puit = conc_cell_surf * 0.33 # nb cellule pour un puit
cell_total = nb_cell_par_puit * nb_puit_cellule
Cf = nb_cell_par_puit / (Vol_depos / 1000) # cellule / mL
Vf = round(Ci * Vi / Cf, 2) # mL

def to_latex(value):
    fstr = f"{value:.2E}"
    val, power = fstr[:4], fstr[-1]
    return val, int(power)

\end{pycode}

%\input{\PATH{protocole}{Comptage_cellules.tex}}

\begin{pycode}
plate_scheme = [
	("A[1-6]", "Filtre"),
	("B[1-6]", "NT"),
	("C[1-6]", "LPS"),
	("D[1-6]", r"\IL"),
	("A-D[1-6]", "-Vera")
]
\end{pycode}

%\input{\PATH{protocole}{Transwell_depos_cellule.tex}}

%\input{\PATH{protocole}{Mesure_TEER.tex}}

\begin{pycode}

nb_de_puits_par_condition = 12 #
vol_puit = Vol_depos # uL

# LPS
LPS = True 
Cf_lps = 10 # ug/mL
Ci_lps = 1000 # ug/mL
Vf_lps = round(nb_de_puits_par_condition * vol_puit * 1.10, 1) # uL + 10 % pour pertes
Vi_lps = round(Cf_lps * Vf_lps / Ci_lps) # uL
Venleve_lps = round(Vf_lps - Vi_lps)

# TNF
TNF =False
Cf_tnf = 0.003 # ug/mL
Ci_tnf = 1 # ug/mL
Vf_tnf = round(nb_de_puits_par_condition * vol_puit * 1.10, 1) # uL + 10 % pour pertes
Vi_tnf = round(Cf_tnf * Vf_tnf / Ci_tnf, 4) # uL
Venleve_tnf = round(Vf_tnf - Vi_tnf, 4)

#IL1
IL1 = True
Cf_il1 = 0.050 # ug/mL
Ci_il1 = 1 # ug/mL
Vf_il1 = round(nb_de_puits_par_condition * vol_puit * 1.10, 1) # uL + 10 % pour pertes
Vi_il1 = round(Cf_il1 * Vf_il1 / Ci_il1, 4) # uL
Venleve_il1 = round(Vf_il1 - Vi_il1, 4)

\end{pycode}

%\input{\PATH{protocole}{Milieu_inflammation.tex}}

\begin{pycode}
start_time_inflammation = "12h50"
start_day_inflammation = 6

p1 = [
	("A[1-6]", "Filtre"),
	("B[1-6]", "NT"),
	("C[1-6]", "LPS"),
	("D[1-6]", r"\IL"),
	("A-D[1-6]", "-Vera")
]
p2 = [
	("A[1-6]", "Filtre"),
	("B[1-6]", "NT"),
	("C[1-6]", "LPS"),
	("D[1-6]", r"\IL"),
	("A-D[1-6]", "-Teno")
]

plates = [("Schéma de l'inflammation des cellules pour le Vérapamil", p1, "table-inflammation-verapamil"), ("Schéma de l'inflammation des cellules pour le tenoxicam", p2, "table-inflammation-tenoxicam")]

\end{pycode}

%\input{\PATH{protocole}{Inflammation_cellule.tex}}

\begin{pycode}

Dex_Mw = 4
final_concentration_DexFITC = 1 #mg/mL
final_concentration_units_DexFITC = "mg/mL"
masse_weighted_DexFITC = 10.3 
masse_weighted_DexFITC_units = "mg"
final_volume_DexFITC = masse_weighted_DexFITC / final_concentration_DexFITC
final_volume_DexFITC_units = "mL"


\end{pycode}

%\input{\PATH{protocole}{Solution_stock_DexFITC.tex}}

%\input{\PATH{protocole}{Dilution_courbe_calibration.tex}}

%\input{\PATH{protocole}{Test_permeabilite.tex}}

\section{Résultats}

\section{Discussions}

\subsection{Commentaires durant l'expérience}


%%%%%%%%%%%%%%%%%%%%%%%ANNEXES%%%%%%%%%%%%%%%%%%%%%%%
\newpage
\section{Annexes}


\printnomenclature


\end{document}