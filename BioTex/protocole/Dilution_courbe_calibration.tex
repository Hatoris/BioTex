\subsection{Préparation de la courbe de calibration}

\begin{enumerate}
\item Prendre la solution stock concentré à 2 mg/mL
\item Réaliser les dilutions en suivant le tableau \ref{Courbe-calibration}
\item Dans des Eppendorf anotté mettre le bon volume de diluant
\item Réalisé les dilutions en cascades en commencant toujours par la concentration la plus petite
\begin{table}[H]
\caption{Dilution pour réalisé la courbe de calibration du Dextran-FITC}
\label{Courbe-calibration}
\begin{pycode}

from tabulate import tabulate
infos = [
    ["Concentration initial ($\mu g/mL$)", "Volume initial ($\mu L$)", "Concentration final ($\mu g/mL$)", "Volume diluant ($\mu L$)",  "Volume Total ($\mu L$)"],
    [2000, 62.5, 1250, 37.5, 100],
    [2000, 37.5, 750, 62.5, 100],
    [2000, 12.5, 250, 87.5 , 100],
    [1250, 10.0, 125, 90.0, 100],
    [750, 10.0, 75, 90.0, 100],
    [250, 10.0, 25, 90.0, 100],
    [125, 10.0, 12.5, 90.0, 100],
    [75, 10.0, 7.5, 90.0, 100],
    [25, 10.0, 2.5, 90.0, 100],
    [12.5, 10.0, 1.25, 90.0, 100],
    [7.5, 10.0, 0.75, 90.0, 100],
    [2.5, 10.0, 0.25, 90.0, 100]
]
table = r"\resizebox{\textwidth}{!}{" + tabulate(infos, headers="firstrow", tablefmt="latex_raw") + "}"
print(table)
\end{pycode}
\label{table-ensemencement}
\end{table}

\item Entouré d'aluminium les eppendorf et les mettre au frigo
\end{enumerate}