\subsection{Préparation des transwells}

Chaque puits sera préalablement revêtis avec du collagen-I à une concentration final de 5~$\mu g/cm^2$. La solution stock de collagen-I est de  3~$mg/mL$. Chaque puits à une surface de 0,33~$cm^2$. Il nous faut donc \py{Q_coll_par_puit}~$\mu g$ de collagen-I par puits.


Nous allons préparer une solution de collagen-I à \py{Cf* 1000}~$\mu g/mL$. Sachant que nous avons \py{nb_puits} puits il nous faut donc \py{Vf} $mL$ de la solution de collagen. Préparer la solution de collagen-I selon le tableau \ref{table-collagen}. Nous déposerons ensuite \py{vol_par_puit * 1000}~$\mu L$ par puits.


\begin{table}[H]
\caption{Dilution de la collagen-I}
\resizebox{\textwidth}{!}{\begin{tabular}{cccc}
\hline
   Concentration final collagen-I ($\mu g/mL$) &   Volume initial ($\mu L$) &   Volume PBS ($\mu L$) &   volume final ($\mu L$) \\
\hline
 \py{Cf * 1000} &                     \py{Vi}  &                  \py{Vf-Vi} &                       \py{Vf} \\
\hline
\end{tabular}}
\label{table-collagen}
\end{table}

Chaque plaque sera préalablement incubées avec la collagen-I : 

\begin{itemize}
\item Début de l'incubation : \py{start_incubation}
\item Fin de l'incubation : \py{end_incubation}
\item TOTAL : \py{TOTAL_incubation}
\end{itemize}
