\subsection{Test de perméabilité des médicaments}

\subsubsection{Avant le début du test}

\begin{enumerate}
\item Sortir 8 plaque receveur 24 puits 
\item Prendre 4 plaques 96 puits pour récolter les prélèvements
\item Dans chaque plaque ajouter 40 $\mu L$ de milieu BMEC dans les puits de 1-12[A-D], c'est puits sont les puits qui recevront les prélèvement des puits donneur qui sont concentrés
%\item Dans chaque plaque ajouter les points de la courbe de calibration, on dépose 100 $\mu L$ de chaque concentration dans les puits 11[A-H], 12A
%\item Dans chaque plaque ajouter le milieu BMEC seul comme blanc dans les puits 12[B-E]
%\item Dans chaque plaque ajouter le milieu contenant 1 mg/mL de Dextran-FITC qui sera déposé sur les cellules dans les puits 12[F-H]  
\end{enumerate}

\subsubsection{Initialisation du test}
\begin{enumerate}
\item 24 h après l'inflammation des cellules, on mesure la TEER (voir \ref{TEER})
\item On prépare le bas d'une plaque 24 puits en ajoutant 600 $\mu L$ de milieu BMEC dans chaque puits \label{milieu-frais-1}
\item On aspire le milieu dans tout les inserts (on commence par les puits contenant que du milieu sans LPS)
\item On déplace les inserts dans la plaque contenant le milieu BMEC frais (étape \ref{milieu-frais-1})
\item On ajoute ensuite dans chaque puits 100 $\mu L$ de BMEC contenant 10 $\mu M$ d'un médicament
\item On ajoute ensuite dans une autre plaque 100 $\mu L$ de BMEC contenant 10 $\mu M$ de l'autre médicament
\item On incube la plaque à $37^\circ C$ +5\% $CO_2$
\item On lance le chronomètre
\item On prépare le bas d'une plaque 24 puits  en ajoutant 600 $\mu L$ de milieu BMEC dans chaque puits \label{milieu-frais-2}
\subsubsection{Premier prélèvement à \textbf{15 min}}
\item On prélève 10 $\mu L$ de chaque puits donneur que l'on dépose dans la plaque en suivant le schéma \ref{table-prelevement-15-30}
\item On déplace l'insert prélevé dans la plaque contenant du milieu frais \ref{milieu-frais-2}
\item Une fois tout les inserts prélevé on remets la plaque à incuber
\item On prélève 100 $\mu L$ de tout les puits receveur que l'on dépose dans la plaque en suivant le schéma \ref{table-prelevement-15-30}
\item On prépare le bas d'une plaque 24 puits  en ajoutant 600 $\mu L$ de milieu BMEC dans chaque puits \label{milieu-frais-3}
\subsubsection{Premier prélèvement à \textbf{30 min}}
\item On prélève 10 $\mu L$ de chaque puits donneur que l'on dépose dans la plaque en suivant le schéma \ref{table-prelevement-15-30}
\item On déplace l'insert prélevé dans la plaque contenant du milieu frais \ref{milieu-frais-3}
\item Une fois tout les inserts prélevé on remets la plaque à incuber
\item On prélève 100 $\mu L$ de tout les puits receveur que l'on dépose dans la plaque en suivant le schéma\ref{table-prelevement-15-30}
\item On prépare le bas d'une plaque 24 puits  en ajoutant 600 $\mu L$ de milieu BMEC dans chaque puits \label{milieu-frais-4}
\subsubsection{Premier prélèvement à \textbf{45 min}}
\item On prélève 10 $\mu L$ de chaque puits donneur que l'on dépose dans la plaque en suivant le schéma \ref{table-prelevement-45-60}
\item On déplace l'insert prélevé dans la plaque contenant du milieu frais \ref{milieu-frais-4}
\item Une fois tout les inserts prélevé on remets la plaque à incuber
\item On prélève 100 $\mu L$ de tout les puits receveur que l'on dépose dans la plaque en suivant le schéma \ref{table-prelevement-45-60}
\item On prépare le bas d'une plaque 24 puits  en ajoutant 600 $\mu L$ de milieu BMEC dans chaque puits \label{milieu-frais-5}
\subsubsection{Premier prélèvement à \textbf{60 min}}
\item On prélève 10 $\mu L$ de chaque puits donneur que l'on dépose dans la plaque en suivant le schéma \ref{table-prelevement-45-60}
\item On déplace l'insert prélevé dans la plaque contenant du milieu frais \ref{milieu-frais-5}
\item Une fois tout les inserts prélevé on remets la plaque à incuber
\item On prélève 100 $\mu L$ de tout les puits receveur que l'on dépose dans la plaque en suivant le schéma \ref{table-prelevement-45-60}

\begin{table}[h]
\caption{Schéma des prélèvement dans la plaque 96 puits pour les temps 15 et 30 min}
\begin{pycode}

plate = BioPlate(12,8) 
plate.set(plate_15_30, merge=True)

table = r"\resizebox{\textwidth}{!}{" + plate.table(tablefmt="latex_raw", stralign="center") + "}"
print(table)
\end{pycode}
\label{table-prelevement-15-30}
\end{table}

\begin{table}[h]
\caption{Schéma des prélèvement dans la plaque 96 puits pour les temps 45 et 60 min}
\begin{pycode}

plate = BioPlate(12,8)
plate.set(plate_45_60, merge=True)

table = r"\resizebox{\textwidth}{!}{" + plate.table(tablefmt="latex_raw", stralign="center") + "}"
print(table)
\end{pycode}
\label{table-prelevement-45-60}
\end{table}

\end{enumerate}