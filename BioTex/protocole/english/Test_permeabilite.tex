\subsection{Permeability assay with the \py{Dex_Mw} kDa Dextran-FITC marker}

\subsubsection{Before starting the test}

This step are explain for one 24 well plates, you may multiply this instruction for each transwell plate you have.

\begin{enumerate}
\item Take 4 24 well plates with no insert in it
\item Write on the plate 15, 30, 45, and 60 min
\item Add 600 $\mu L$ of medium in each wells
\item Put the 4 plates in the incubator 
\item Take 2 96 black well plates for fluorescence reading
\item Add a sticker on each 96 well plates then write 15-30 and 45-60 on them 
\item In each 96 well plates add medium in the wells 1-10[A-D], those wells are use to collect donnor well from the 24 well plates
\item Then in each 96 well plates add calibration curve dilution, put 100 $\mu L$ of each concentration in well 11[A-H] and 12A
\item In each 96 well plates add medium alone as blanck value in wells 12[B-E]
\item Finaly add working solution of \py{Dex_Mw} kDa Dextran-FITC in wells 12[F-H] 
\end{enumerate}

\subsubsection{Starting the permeability test}
\begin{enumerate}
\item Mesure the TEER (see \ref{TEER})
\item Take the 15 min 24 wells plate (after at least 15 min of incubation)
\item Take the 24 wells plate with inserts (cells) on it
\item Move each inserts in the 15 min 24 wells plates
\item Carefully aspirate the medium in each inserts
\item Add 100 $\mu L$ of 1 $mg/mL$ \py{Dex_Mw} kDa Dextran-FITC in each inserts
\item Start the stopwatch  
\item Inubate the plate at $37^\circ C$ +5\% $CO_2$
\subsubsection{First sample at \textbf{15 min}}
\item Take from the incubator the 24 wells plates with cells and the 30 min 24 wells plate
\item We sample 10 $\mu L$ of each donor well and we drop it in the 15-30 96 wells plates following the schema \ref{table-prelevement-15-30}
\item When all inserts are sampled, move all the insert in the 30 min 24 wells plate
\item Put back in the incubator the 30 min 24 wells plate
\item Sample 100 $\mu L$ of all receiver well and put it in the 15-30 96 wells plate following the schema \ref{table-prelevement-15-30}
\subsubsection{Second sample at \textbf{30 min}}
\item Take from the incubator the 24 wells plates with cells and the 45 min 24 wells plate
\item We sample 10 $\mu L$ of each donor well and we drop it in the 15-30 96 wells plates following the schema \ref{table-prelevement-15-30}
\item When all inserts are sampled, move all the insert in the 45 min 24 wells plate
\item Put back in the incubator the 45 min 24 wells plate
\item Sample 100 $\mu L$ of all receiver well and put it in the 15-30 96 wells plate following the schema \ref{table-prelevement-15-30}
\subsubsection{Third sample at \textbf{45 min}}
\item Take from the incubator the 24 wells plates with cells and the 60 min 24 wells plate
\item We sample 10 $\mu L$ of each donor well and we drop it in the 45-60 96 wells plates following the schema \ref{table-prelevement-45-60}
\item When all inserts are sampled, move all the insert in the 60 min 24 wells plate
\item Put back in the incubator the 45 min 24 wells plate
\item Sample 100 $\mu L$ of all receiver well and put it in the 45-60 96 wells plate following the schema \ref{table-prelevement-45-60}
\subsubsection{Fourth sample at \textbf{60 min}}
\item Take from the incubator the 24 wells plates with cells
\item We sample 10 $\mu L$ of each donor well and we drop it in the 45-60 96 wells plates following the schema \ref{table-prelevement-45-60}
\item Sample 100 $\mu L$ of all receiver well and put it in the 45-60 96 wells plate following the schema \ref{table-prelevement-45-60}

\begin{table}[h]
\caption{Sample schema for the 96 well plates at 15-30 min}
\begin{pycode}

from BioPlate import BioPlate
plate = BioPlate(12,8)
value = (
    ("1-5[A-H]", ["100", "200", "300", "400", "Filter"]),
    ("6-10[A-H]", ["100", "200", "300", "400", "Filter"]),
    ("A-D[1-10]", "-D"),
    ("E-H[1-10]", "-R"),
    ("1-5[A-H]", "-15"),
    ("6-10[A-H]", "-30"),
    ("11[A-H]", ["1250", "750", "250", "125", "75", "25", "12.5", "7.5"]),
    ("12[A-H]", ["2.5", "BMEC", "BMEC", "BMEC", "BMEC", "DFITC", "DFITC", "DFITC"]),
)
plate.set(value, merge=True)

table = r"\resizebox{\textwidth}{!}{" + plate.table(tablefmt="latex_raw", stralign="center") + "}"
print(table)
\end{pycode}
\label{table-prelevement-15-30}
\end{table}

\begin{table}[h]
\caption{Sample schema for the 96 well plates at 45-60 min}
\begin{pycode}

from BioPlate import BioPlate
plate = BioPlate(12,8)
value = (
    ("1-5[A-H]", ["100", "200", "300", "400", "Filter"]),
    ("6-10[A-H]", ["100", "200", "300", "400", "Filter"]),
    ("A-D[1-10]", "-D"),
    ("E-H[1-10]", "-R"),
    ("1-5[A-H]", "-45"),
    ("6-10[A-H]", "-60"),
    ("11[A-H]", ["1250", "750", "250", "125", "75", "25", "12.5", "7.5"]),
    ("12[A-H]", ["2.5", "BMEC", "BMEC", "BMEC", "BMEC", "DFITC", "DFITC", "DFITC"]),
)
plate.set(value, merge=True)

table = r"\resizebox{\textwidth}{!}{" + plate.table(tablefmt="latex_raw", stralign="center") + "}"
print(table)
\end{pycode}
\label{table-prelevement-45-60}
\end{table}

\end{enumerate}