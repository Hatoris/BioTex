\subsection{Measure fluorescence with TECAN plate reader}
\begin{enumerate}
\item Entourer les plaques d'aluminium
\item Déamarer le lecteur de plaque et choisir la méthode dexFITC
\item Vérifié que tout les puits sont sélectionés
\item Vérifier que l'exitation est régler sur 485 nm et l'émission sur 535 nm
\item Vérifier que le gain est réglé sur 50 (on peut l'ajuster au besoin si des puits sont saturés, \textbf{MAIS VOUS DEVEZ ABSOLUMENT FAIRE TOUTE LES MESURES AVEC LE MÊME GAIN !}
\item Mesurer la fluorescence pour tout les puits
\item Sauvegarder tout les résultats dans une feuille excel et nom le fichier avec les informations importantes sur l'expérience
\item Sauvegarder les fichiers sous :  \\ Desktop/LabRoullin/Florian/DATE\_usefull\_informations/Date\_usefull\_informations.xlsx
\item Prendre une clef USB, s'envoyer les résultats par courriels ou via 0365.umontreal.ca
\end{enumerate}
